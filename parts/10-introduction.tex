%======================================================================
\chapter{Introduction}
%======================================================================
As smart power grids become more prevalent and data on smart building energy usage becomes more available, the opportunity for energy efficiency increases drastically. Data collected from smart buildings allows a neural network to control the workings of a building in an efficient manner. We propose to expand on the loads that can be controlled through BEMOSS by enabling control of a DC motor. Building energy management open source software (BEMOSS) is an internet of things (IoT) software that was developed at Virginia Tech. This proposed control of a DC motor through an IoT software presents the opportunity to close blinds, open barriers, and move objects throughout the building. The ability to open and close blinds building wide permits the ability to regulate the interior temperature of a building with a considerably lower power consumption. For instance, closing blinds during a hot day will naturally cool a room while opening blinds during a cold day will naturally heat the room. This added ability to control the interior environment will reduce the energy consumed by a heating ventilation air conditioning (HVAC) system. 
\todo[inline]{Add networking info and controlling python scripting on BEMOSS}
%----------------------------------------------------------------------
\section{Background Study}
%----------------------------------------------------------------------
Authors in~\cite{Han2014} proposed a energy management system based around Zigbees and PLCs. Han's system would place a energy measurement and communication unit in each outlet and light in the consumer's home. The energy usage will be measured and the data collected will be sent to a home server run on a Zigbee. These Zigbees will analyze the data and give feedback to the user on how to better manage their energy usage. Renewable energy will be connected to a PLC to allow the use of renewables as they need to converted to AC. The home server on the Zigbee will predict the amount of energy that will be obtained by renewables by accessing weather data and automatically adjust the user's device schedule.

In~\cite{Collotta2015} the authors once again proposed sensors in all outlets and lights to measure the energy usage in homes. Renewable energy sources, like solar panels and wind energy, are connected to an inverter and battery system, to allow the storage of excess power. Collotta connects the internet of things devices using Bluetooth rather than WiFi, due to the lower power consumption. The system checks the current time and cross checks it with the peak time for energy consumption by the user's energy provider. If it is during peak times, the system checks the battery system for excess stored energy. When the user is trying to use energy during a peak time and without a battery charge the system warns the user, but allows the user to ignore these warnings.

In~\cite{Mantovani2014} the energy system uses two model predictive controllers (MPC), one for the building's HVAC system and one for the system's battery power flow. The building's energy management system predicts the temperature of each room separately using sensors and a Kalman filter for robustness. The battery system put in place by Mantovani can run on two modes, one to minimize energy cost and another to minimize power flow. The building is also c-onnected to wind turbines and photo-voltaic cells and predicts the energy produced using a simplified model.

The authors in~\cite{Hannan2018} propose a Internet of Energy network rather than an IoT network. An IoE combines IoT and smart grid technology using four key components: Energy Router, Storage System, renewable sources, and plug-and-play appliances. The IoE allows for an easier way to produce a net zero energy building, that produces as much or more energy than it consumes. The energy router consists of a solid-state-transformer, grid control, and communication meant for data management. The storage system like batteries, reduce the stress on the power grid and lower voltage fluctuation. Renewable sources reduce carbon emmisions, however they reduce harmonics that need to be handled with additional hardware. Lastly the plug-and-play appliances are the devices that the end-user uses in a home.

In~\cite{Pan2015} the system heavily interfaces with the users' smartphones as a way to monitor the building occupants. Since smartphones almost all have a way to track GPS, the system tracks the users' location and send it to the building's server. The building's server is broken up into a number of subservers to handle an individual room's needs. By tracking location the building can do such things like pre-heating or pre-cooling a room before the user is even in the building. All the subservers are connected to the main server which is connected to cloud storage which is used for hosting the large amount of data and handling the intense computations.


\section{Project Statement}
Our project had three main goals. The first goal was to implement a new device not currently supported by BEMOSS on BEMOSS. The second goal was to run BEMOSS on a single board computer, such as a raspberry pi. The last goal is to develop a control algorithm to reduce energy cost and implement the algorithm on BEMOSS.

\section{Report Structure}

%%% Local Variables:
%%% mode: latex
%%% TeX-master: "../finalReportMainV1"
%%% End:
